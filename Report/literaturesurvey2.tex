\chapter{Literature Survey}
An aspect of Cloud Computing is the illusion of infinite computing resources available on demand, thereby eliminating the need for Cloud Computing users to plan far ahead for provisioning. For this, realizing the economies of scale afforded by statistical multiplexing and bulk purchasing requires the construction of extremely large data-centers. Building, provisioning, and launching such a facility is a hundred-million-dollar undertaking \cite{Berkeley}.\\[0.2cm]
A system composed of a virtual network of virtual machines capable of live migration across multi-domain physical infrastructure have been constructed. By using dynamic availability of infrastructure resources and dynamic application demand, a virtual computation environment is able to automatically relocate itself across the infrastructure and scale its resources \cite{SRAS}. Thus the QoS improvements can be met using this virtual machine setup. Depending on the type of application, the generated workload can be a highly varying process that turns difficult to find an acceptable trade-off between an expensive over-provisioning able to anticipate peak loads and a sub performing resource allocation that does not mobilize enough resources. These properties can be leveraged to derive a probabilistic assumption on the mean workload of the system at different time resolutions \cite{DRMC}.\\[0.2cm]
There are many proposals that dynamically manage Virtual Machines by optimizing some objective function such as minimizing cost function, cost performance function and meeting QoS objectives. The objective function is defined as Utility property which is selected based on measures of response time, number of QoS, targets met and profit etc. \cite{SRAS}. This Utility property can then be adjusted to include factors of individual importance. They have been modified to meet needs according to current demand. But they may prove complex and  may not work in a practical virtualization cloud system with real workload. These approaches work best for stationary demands and may not give optimal solution for dynamic resource requirements. The utility property can be modified to include dynamic demands and allocation and hence has to be formulated accordingly.
