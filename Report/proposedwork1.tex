\chapter{Proposed Work}
The problem, till now, has been addressed by scheduling one job at a time to the best server and thus following the same strategy for a group of jobs. The problem can also be viewed as scheduling a group of jobs to the best servers at one instant of time. This approach has been taken because it may not be necessary that the best allocation for each job is the best for the given group of jobs i.e., it is possible to have a collective best outcome which may be better than a collection of individual best outcomes. Hence this problem can be looked at from the technique of game theory which fundamentally states the same.\\[0.2cm]
Game theory is the study of strategic decision making. Strategies and payoffs are calculated and a payoff matrix is formed which will contain the values obtained on applying a utility function. This approach has been widely used in various fields of computer science. The matrix given below can be obtained for load balancing problem where each job can have different affinity towards different servers.
\begin{center}
    \begin{tabular}{ | l | l | l | l |}
    \hline
    \small {Job/Server} & $S_1$ & $S_2$ & $S_3$\\ \hline
    $J_1$ & 5 & 2 & 5\\ \hline
    $J_2$ & 9 & 1 & 3\\ \hline
    $J_3$ & 4 & 1 & 8\\ \hline
    \end{tabular}
\end{center}
Such kind of representation has motivated us to give a game-theoretic approach to this problem. In the coming months, the objective will be to study game theory and to use it to obtain optimal solution for load balancing in Cloud computing.


