\chapter{Introduction}
Cloud computing refers to both the applications delivered as services over the Internet and the hardware and systems software in the data centers that provide those services. Cloud service delivery is divided into three models. The three service models are \cite {Ratan} :

\section{Cloud Software as a Service (SaaS)}
The capability provided to the consumer is to use the providers applications running on a cloud
infrastructure. The applications are accessible from various client devices through a thin client
interface such as a web browser. 

\section{Cloud Platform as a Service (PaaS)}
The capability provided to the consumer is to deploy onto the cloud infrastructure, consumer
created or acquired applications created using programming languages and tools supported by the
provider. The consumer does not manage or control the underlying cloud infrastructure but has
control over the deployed applications and possibly application hosting environment
configurations.

\section{Cloud Infrastructure as a Service (IaaS)}
The capability provided to the consumer is to provision processing, storage, networks, and other
fundamental computing resources where the consumer is able to deploy and run arbitrary
software, which can include operating systems and applications. The consumer does not manage
or control the underlying cloud infrastructure but has control over operating systems, storage,
deployed applications.\\[0.2cm]
Load balancing is used to make sure that none of the existing resources are idle while others are being utilized. To balance load distribution, migration of the load from the source nodes (which have surplus workload) to the comparatively lightly loaded destination nodes can be done.

\section{Problem Statement}
\begin{itemize}
\item A study of the existing load balancing algorithms in Cloud Computing and
implement a new hybrid load balancing algorithm. This will include an 
analysis of these above algorithms and to draw conclusions based on 
their execution time. 
\end{itemize}

