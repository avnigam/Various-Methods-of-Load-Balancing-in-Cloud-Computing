\vspace{-1in}
\chapter{Introduction}
Cloud computing is a model for enabling ubiquitous, convenient, on-demand network access to a shared pool of configurable computing resources (e.g., networks, servers, storage, applications, and services) that can be rapidly provisioned and released with minimal management effort or service provider interaction \cite{NIST}. This model enables users to access supercomputer-level computing power elastically on an on-demand basis, freeing the users from the expense of acquiring and maintaining the underlying hardware and software infrastructure and components. ``A Cloud is a type of parallel and distributed system consisting of a collection of interconnected and virtualized computers that are dynamically provisioned and presented as one or more unified computing resources based on service-level agreements established through negotiation between the service provider and consumers" \cite {NIST}. Cloud computing, as a current commercial offering, started to become apparent in late 2007. It was intended to enable computing across widespread and diverse resources, rather than on local machines or at remote server farms. Where these clusters supply instances of on-demand Cloud computing; provision may be comprised of software (e.g. Software as a Service, SaaS) or of the physical resources (e.g. Platform as a Service, PaaS) \cite {Randles1}.\\[0.2cm]
Load Balancing is the process of distributing the load among various nodes of a distributed system when it becomes difficult to predict the number of requests that will be issued to a server. It considers factors like execution time, resource availability and requirement among others to improve job response time, throughput, etc. In order to provide better service-level agreements, the cloud provider has to provide such improvements to the user. Load Balancing is a method to distribute workload across one or more servers, network interfaces, hard drives, or other computing resources.

\section{Report Organization}
The report consists of two phases- different load balancing algorithms and a new approach to load balancing.\\[0.2cm]
In Phase I, a basic introduction on various cloud service models has been provided, followed by a literature survey on the existing load balancing algorithms, their advantages and limitations. The phase then continues onto a detailed description of the work done in studying, implementing and analysing three load balancing algorithms. This also includes a hybrid algorithm that has been proposed by us. In the last part of this phase, the basic idea for a new approach along with its motivation have been recorded.\\[0.2cm]
In Phase II, one of the problems in Cloud Computing has been tackled in a new way, the illusion of infinite resources, considering load balancing factors. An introduction on this problem and the Quality of Service protocols that can be achieved has been described. A literature survey has been done on the various approaches of this problem . In the later part of this phase, the design, implementation and analysis of the proposed approach has been explained in detail followed by the limitations of this approach.\\[0.2cm]


