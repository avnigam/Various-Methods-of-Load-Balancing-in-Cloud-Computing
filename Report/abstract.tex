\vspace{2in}
\begin{abstract}
As Cloud Computing is an emerging field, many improvements are being proposed to provide users with better services and facilities. This work deals with the illusion of infinite resource availability on demand, one of the new aspect in Cloud Computing. Initially, few proposed algorithms for Cloud Computing have been implemented to understand load balancing in cloud computing and a new hybrid algorithm has been proposed. Further, a combination of forecasting models and game theoretic approaches have been proposed so as to continue providing this illusion without any glitches. This implementation is a new way of looking at the problem and with coordination from different providers it becomes possible for each provider to decide his best strategy. This work provides an efficient way for the cloud provider to decide on his strategies to execute a job i.e., whether to use his own services to execute (self-execute) or to pay rent for the services to other cloud providers. Forecasting has been done to get an idea of previous demands. Game theory is a concept that is generally applied for economical undertakings, generally for the current period, and is a good way to engage it in deciding the strategies adopted by an organization. Hence the design considers both the previous as well as current demand to decide on the provider's strategy making the results more accurate. This work combines two different approaches and based on their results decides upon a strategy that has minimum deviation. The results obtained from this utility function show an almost equal distribution of Rent and Self-execute strategies. This work can be enhanced by including more factors, especially of financial importance, in the utility and providing methods for scalability.
\end{abstract} 
