\chapter{Conclusion}
In this work, we have discussed one approach of how the illusion of infinite resources in Cloud Computing can be further optimized without incurring any loss for the cloud provider. QoS can be improved and thus it provides the user with better facilities. The cloud provider can decide whether he wants to pay rent for the load and prevent disappointing the user or just execute only what is possible. Forecasting followed by game theory gives a better approach of including the data available till the current point and will help in correctly deciding the strategy. From these results, a cloud provider can also check the number of times he is paying rent for the load and based on further calculations he can decide to own more facilities such that the frequency of renting will decrease and he may earn more profit. This can be modelled as a Ski-Rental problem and further worked out.\\[0.2cm]
This work provides one method of tackling the problem and it further opens up interesting avenues for improvement. The utility property can be changed to suit the agreement between the provider and the user. This approach considers execution time as one of the factors and this may change depending on the network traffic and other dynamic factors which will further complicate the utility property. There can be different utility functions that can be used considering more parameters. Other game-theoretic concepts and forecasting models can also be used to obtain the results. Since the avenues are deep and the domain is still growing, updates on this problem will keep on increasing until an optimal solution is reached.
